
\documentclass[11pt, a4paper]{article}
\usepackage{graphicx}
\usepackage{amsmath}
\usepackage{listings}


\title{Assignment No 3:Fitting data to models}
\author{G Ch V Sairam , EE19B081} 
\date{03-03-2021}
\begin{document}		
		
\maketitle 
\section{Introduction}

This week's assignment starts off with generating data as a linear combination
of the Bessel Function and some noise.
f(t)=A*$J_2$(t)-B*t+n(t)
where
\begin{itemize}
  	\item A=1.05
  	\item B=-0.105
  	\item $J_2$(t)=Bessel function of 2nd order
  	\item n(t)= Noise function
\end{itemize}
  
\section*{Assignment problems}
\subsection*{Part 2}
loadtxt function was used to import data from the file.
The data consists of 10 columns. The first columns represents the time and the remaining 9 columns are the required function values with varying levels of noise.
\lstinputlisting[language=Python, firstline=23, lastline=25]{EE19B081_assign3.py}
\subsection*{Parts 3,4}
The plot generated from the data given and the original function with A=1.05 and B=-0.105 are plotted using the plot function.

   \begin{figure}[!tbh]
   	\centering
   	\includegraphics[scale=0.5]{plot0}   
   	\caption{Combined plot for parts 3 and 4}
   	\label{fig:plot0}
   \end{figure} 

\subsection*{Part 5}
The errorbars are represented by red dots
\begin{figure}[!tbh]
	\centering
	\includegraphics[scale=0.5]{plot1}
	\caption{Plot for part 5}
	\label{fig:plot1}
\end{figure}

\subsection*{Parts 6 and 7}
We an use the if function and use flags to check whether the 2 arrays are equal or not. And we can find the error matrix by appending the values using a for loop.

\subsection*{Part 8}
\lstinputlisting[language=Python, firstline=96, lastline=102]{EE19B081_assign3.py}
The contour plot of the mean squared error versus the parameters A and B
is given in the below plot. As we can see, there is a single minima at the indicated point in the plot.

\begin{figure}[!tbh]
	\centering
	\includegraphics[scale=0.5]{plot2}
	\caption{Plot for part 8}
	\label{fig:plot2}
\end{figure}

\subsection*{Part 10}
The plot of error in the estimate of A and B with noise is given in figure 4.
As we can see, the error in estimate has non linear variation.
\begin{figure}[!tbh]
	\centering
	\includegraphics[scale=0.5]{plot3}
	\caption{Plot for part 10}
	\label{fig:plot3}
\end{figure}

\subsection*{Part 11}
Here , we must use the plot.loglog function instead of just using plot.
However when we plot the error returned by the lstsq function It is nearly linear. This is because our noise varies on a logarithmic scale.
\begin{figure}[!tbh]
	\centering
	\includegraphics[scale=0.5]{plot4}
	\caption{Plot for part 11}
	\label{fig:plot4}
\end{figure}

\end{document}
